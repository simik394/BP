\chapter*{Úvod}
\addcontentsline{toc}{chapter}{Úvod}

Správa úkolů není relevantní pouze pro velké firmy, naopak představuje důležitou dovednost pro každého, kdo chce dosáhnout svých cílů a úspěchů.. Nabízí se celá řada nástrojů na podporu efektivního doručování hodnoty, které pomáhají udržovat přehled o tom co je třeba udělat kdy to udělat a jak to udělat. 

V této práci se budu zabývat možnostmi využití některých nástrojů, které jsou běžně používány v podnikové sféře, pro potřeby skautského oddílu. Zajímá mě, jak by tyto nástroje fungovaly v prostředí, kde je důležitá spolupráce a sdílení informací mezi členy, kteří ale zároveň nejsou ke své činnosti podporování žádným monetárním obnosem. Jejich motivací pak obvykle bývá jen chuť strávit nějaký dobrý čas s kamarády, nebo předat část z toho co sami ve svém dětství obdrželi. Není proto problém s jejich motivací pro plnění hlavních procesů přinášejících hodnotu. Zato v případě procesů, které zajišťují řízení a koordinaci spolupráce, nezbytných pro přípravu komplexnější událostí, jako například tábor. Jehož příprava začíná již v březnu, nicméně vždy je nutné do vymýšlet programy na posledních chvíli před jejich naplánovaným začátkem. To je přirozeně nežádoucí stav, který má šanci vyústit v nedostatek času, materiálů nebo nápadů, Což logicky není cestou ke kvalitnímu programu, který je naším cílem. 

Když jsme u cílů, bylo by vhodné zmínit, že nejsme zcela nezávislá jednotka. Jsme součástí větší organizace s hierarchickou strukturou, v čele s orgány jako náčelnictvo, nebo výkonná rada. Ve snaze nastavit určité standardy kvalitě, byla v roce 2015 vypracována strategie do roku 2022. Aktuálně se již pracuje na nové, do roku 2030, jenž má být kompletní přibližně za rok. Není však kam spěchat, jelikož pokud budeme sledovat důsledky minimalistických přístupů k řízení, které se ve skautingu v důsledku jeho dobrovolnosti nutně vyskytnou. Zjistíme, že vlastně nemáme k dispozici ani žádnou aktivně a systematicky sledovanou metriku, podle které by bylo možné objektivně určit, zda se nám alespoň daří udržovat kvalitu programů na nezhoršující se úrovni. Takže stěží můžeme splnit, byť jen první doporučení ze strategie, které stanovuje cíl pro skauty, být odborníky na výchovu. Navíc, by vzhledem k [výzkum na org náročnost v závislosti na počtu]. Zohlednit i skutečnost, kterou představuje navýšení počtu členů za posledních 10 let o více než 50 procent, na současných téměř 69 tisíc.  

Je faktem, že i oddíl v jehož vedení se podílím, je s aktuálním stavem více než 15 lidí, mezi které se rozloží práce nutná k provozu tábora. Značně odlišný od roku 2008 kdy jsem začínal chodit a byli jsme vedeni tříčlenným týmem. Současný stav samozřejmě odemyká řadu možností, avšak pouze za předpokladu vhodného řízení je bude možné využít v plném rozsahu. Pokud by se tak nějaký vedoucí obětoval a vzal si tuto roli manažera na starosti. Stálo by to vůbec za to? Nenarušili by se kamarádské vztahy mezi vedoucími? A i kdyby to bylo jen za cenu zkaženého zážitku pro jednoho, není i to příliš? Kolik by to vůbec vyžadovalo času a práce? Které z dostupných nástrojů má smysl zkusit? Stačil by na to jeden vedoucí, pokud by se jeho časová zátěž neměla nijak citelně zvýšit oproti běžné činnosti? A jak by se změnila odpověď na tyto otázky v případě, že by oddíl měl svoje procesy (stávající i potenciálně přidané) integrované s automatizovaným systémem přizpůsobeným jejich potřebám? V této práci se budu snažit získat odpovědi na poslední 4 z těchto uvedených. 

Hlavním cílem této práce jsem proto zvolil porovnání nároků na zdroje při implementaci a provozu několika vybraných, běžné používaných prostředků, ke zvýšení efektivity vykonávané práce, skrze podporu úkolů v průběhu celého jejich životního cyklu, v závislosti na využití automatizace, na modelovém případu našeho oddílu. 

Vzhledem k současnému nedostatečnému stavu řízení, bude teoretická část této práce věnována analýze výchozího stavu. Začínaje vymezením možností podpory životního cyklu úkolů. Následováno zmapováním oddílových zdrojů, procesů I cílů. Získané hodnoty budou na konci této časti použity jako východisko, pro identifikaci relevantních optimalizací, které představují hlavní výstup teoretické části. 

Vycházejíce z výstupu teoretického základu, navazuje praktická část nejdříve identifikací Nejnutnějších opatření potřebných pro možný začátek efektivního a řízení. Následují kapitoly obsahující návrh zaprvé automatizace stávajícího řešení s implementovanými nejnutnějšími opatřeními. Následovaný sestavením několika odlišných softwarových řešení zaměřených buď na podporu kvality dokončených úkolů, snížení jejich náročnosti, nebo obojího. V poslední kapitole pak vyhodnotím náročnosti jednotlivých řešení ve fázi implementace, poté i provozu. Náročnost v užitné fázi bude posuzována ve dvou variantách. Kdy první představuje užití systému bez Automatizace, Zatímco v druhém případě bude snaha o maximalizaci míry automatizace. 
